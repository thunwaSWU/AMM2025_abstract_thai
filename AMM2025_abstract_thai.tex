%====================================================
% MUST COMPILE WITH XeLaTeX
% Put font files in the same folder as the .tex file
%====================================================
% Abstract Template (THA)
% The 29th Annual Meeting in Mathematics (AMM 2025)
% Srinakharinwirot University
%====================================================

%!TEX TS-program = xelatex
%!TEX encoding = UTF-8 Unicode

\documentclass[12pt, a4paper]{article}
\usepackage{geometry,graphicx,tikz}
\usepackage{amssymb,amsmath,amsthm}
\usepackage{fancyhdr}

\geometry{top=5cm, bottom=2.5cm, left=2.5cm, right=2.5cm, headheight=3cm}

%%%%%%%%%%%%%%%%%%%%%%%%%%
% SETUP FOR THAI LANGUAGE
%%%%%%%%%%%%%%%%%%%%%%%%%%

\usepackage{polyglossia}
\usepackage{fontspec}
\usepackage{xltxtra}
\usepackage{xunicode}
\XeTeXlinebreaklocale "th"
\XeTeXlinebreakskip = 0pt plus 0pt 
\defaultfontfeatures{Mapping=tex-text} 

\newfontfamily\thaifont{THSarabunNew.ttf}[
    Path,
    BoldFont= THSarabunNew Bold.ttf,
    ItalicFont=THSarabunNew Italic.ttf,
    BoldItalicFont=THSarabunNew BoldItalic.ttf,
    Scale = 1.35]
\newfontfamily\thaifonttt{Sarabun-Regular.ttf}[
    Path,
    Script = Thai,
    Scale = 0.95]
\setmainlanguage{thai}
\renewcommand{\baselinestretch}{1.3}

\usepackage[Latin,Thai]{ucharclasses}
\setTransitionsForLatin
  {\begingroup\rmfamilylatin}
  {\endgroup}
\usepackage[hidelinks]{hyperref}

\newenvironment{AMM-abstractTH}[4][]{
  \begin{center}
    { \renewcommand\textsuperscript[1]{}\par}
    {{\Large\bfseries #2}\par}
    \medskip
    {\large #3\par}
    \bigskip
    {\small #4\par}
    \bigskip\bigskip
    {{\large\bfseries บทคัดย่อ}\par}
  \end{center}
}{ 
    \bigskip
    \hrule
    \bigskip
}
\newcommand{\mykeywordsTH}[1]{%
    \noindent \textbf{คำสำคัญ:} #1 \par
}
\newcommand{\myMSC}[1]{
    \noindent \textbf{2020 MSC:} #1 \par
}

\pagestyle{fancy}
\renewcommand{\headrulewidth}{0pt}

\AddToHook{begindocument/end}{\renewcommand{\thefootnote}{\fnsymbol{footnote}}}

\fancyhead[L]{\footnotesize
The 29\textsuperscript{th} Annual Meeting in Mathematics (AMM 2025)\\
Department of Mathematics, Faculty of Science,\\
Srinakharinwirot University,
Bangkok, Thailand}

%%%%%%%%%%%%%%%%%%%%%%%%%%%%%%%%%%%%%%%%%%%%%%%%%%%
%%%%% DO NOT make any changes above this line %%%%%
%%%%%% DO NOT use any customized definitions %%%%%%
%%%%%%%%%%%%%%%%%%%%%%%%%%%%%%%%%%%%%%%%%%%%%%%%%%%

\begin{document}

%%%%%%%%%%%%%%%%%% INSTRUCTION %%%%%%%%%%%%%%%%%%%%
% Use the following command to write your abstract:
%
% \begin{AMM-abstractTH}[]
% {Title}
% {Authors (use \textsuperscript as institution markers)}
% {Institutions (use \textsuperscript as institution markers)}
% Abstract text
% \end{AMM-abstractTH}
%
%%%%%%%%%%%%%%%%%%%%%%%%%%%%%%%%%%%%%%%%%%%%%%%%%%%

\begin{AMM-abstractTH}[]
{ชื่อเรื่อง} %TITLE
{ชื่อผู้แต่งคนแรก\textsuperscript{1,}\footnote{ผู้นำเสนอ (name@email.com)}, ชื่อผู้แต่งคนที่สอง\textsuperscript{1}
และ ชื่อผู้แต่งคนที่สาม\textsuperscript{2,}\footnote{(ตัวอย่าง) ได้รับทุนสนับสนุนจาก...}} %AUTHORS
{\textsuperscript{1}ภาควิชาคณิตศาสตร์ คณะวิทยาศาสตร์
มหาวิทยาลัยศรีนครินทรวิโรฒ 10110\\ \smallskip
\textsuperscript{2}ภาควิชาคณิตศาสตร์ สถิติและคอมพิวเตอร์ คณะวิทยาศาสตร์ \\ มหาวิทยาลัยอุบลราชธานี 34190} %AFFILIATIONS

%YOUR ABSTRACT GOES HERE
บทคัดย่อภาษาไทยไม่ควรเกิน 250 คำ 
โดยให้ใช้แบบและขนาดอักษรของข้อความต่าง ๆ ดังในไฟล์นี้ 
และไม่มีการอ้างอิงในบทคัดย่อ แนะนำให้ผู้เขียนใช้ลำดับการเขียนบทคัดย่อดังนี้  

1) ความเป็นมาและความสำคัญของปัญหา: วางคำถามในบริบทกว้าง ๆ และเน้นวัตถุประสงค์ของการวิจัย 

2) วิธีการดำเนินการวิจัย: อธิบายวิธีการหลักที่ใช้ในงานวิจัยสั้นๆ 

3) ผลการวิจัย: สรุปผลการค้นที่พบหลัก ๆ ของบทความ และ 

4) สรุปผลการวิจัยและอภิปรายผล: ระบุข้อสรุปหลักหรือการตีความ 

บทคัดย่อควรแสดงวัตถุประสงค์ของบทความแต่ต้องไม่มีผลที่ไม่ได้นำเสนอและไม่ควรสรุปเกินจริง
ดูข้อมูลเพิ่มเติมเกี่ยวกับระบบการจำแนกวิชาคณิตศาสตร์ (2020 MSC) ได้ที่ \url{https://mathscinet.ams.org/msnhtml/msc2020.pdf}
\end{AMM-abstractTH}

%%%%%%%%%%%%%%%%% Keywords and MSC %%%%%%%%%%%%%%%%%%%%%%

\mykeywordsTH{คำสำคัญ 1, คำสำคัญ 2, คำสำคัญ 3} %Please include 3-5 keywords here. Use comma to separate items in the list.
\smallskip
\myMSC{nnXxx, nnXxx, nnXxx} %Please include MSC here. The first item is the primary MSC. Use comma to separate items in the list.

\end{document}